\documentclass[12pt, letter]{article}
\usepackage[margin=0.8in]{geometry}
\usepackage[T1]{fontenc}
\usepackage{fourier}
\usepackage{titling}
\usepackage{bm}
\usepackage{amssymb}
\usepackage{amsmath}

\setlength{\droptitle}{-5em} 
\usepackage[parfill]{parskip}

\author{Zach Neveu}
\title{ Homework 8: Estimation Review }

\begin{document}
\maketitle
\section{Least Squares}
\[
\hat{\theta}_{LS} = \frac{1}{\bm{g'g}}\bm{g'y}
\] 
\section{MVUB}
\begin{align}
\hat{\theta}_{MVUB} = \frac{1}{\bm{g'C_z^{-1}g}}\bm{g'C_z^{-1}y} = \frac{1}{ g'C_z^{-1}g}g'C_z^{-1}[g\theta+z]
= \theta + \frac{1}{g'C_z^{-1}g}g'C_z^{-1}z\\
Var(\theta) = E\{|\theta - \hat{\theta}|^2\} = E\{|c'z|^2\} = E\{(c'z)(c'z)^*\} = \bm{c'C_zc} = \frac{1}{\bm{g'C_z}^{-1}\bm{g}}
\end{align}

\section{Linear MMSE}%
\label{sec:linear_mmse}

\begin{gather*}
\hat{\theta}_{lin.MMSE} = \frac{R_\theta}{1+R_\theta \bm{g'C_z^{-1}g}}\bm{g'C_z^{-1}y} \\
MMSE = \bm{R}_\theta - \bm{r}'\bm{R}^{-1}\bm{r} = \frac{R_\theta}{1+R_\theta g'C_z^{-1}g}
\end{gather*}
\[
r = E\{\bm{y}\theta\} = E\{[\bm{g}\theta+\bm{z}]\theta^*\} = \bm{g}E\{\theta^2}\} + 0
\] 

\section{ML Estimate}%
\label{sec:ml_estimate}

If $\bm{Z}\sim \mathcal{N}(0, \bm{C_z})$, then $\hat{\theta}_{ML} = \hat{\theta}_{MVUB}$
\[
\hat{\theta}_{ML} = \hat{\theta}_{MVUB} = \frac{1}{\bm{g'C_z^{-1}g}}\bm{g'C_z^{-1}y}
\] 

\section{ML Detection of $\theta$}%
\label{sec:ml_detection_of_theta_}
This problem can be thought of in two steps. The first step is to determine the ML estimate of theta as in section \ref{sec:ml_estimate}. The second step is to classify our estimate of $\theta$ as either $\theta_0$ or $\theta_1$. From section \ref{sec:ml_estimate}, we know that our estimator for $\theta$ is unbiased. This reduces the detection problem to simply setting a threshold at $\gamma = \frac{\theta_0 + \theta_1}{2}$. Assuming $\theta_0>\theta_1$, the detection problem can be stated as
\[
\hat{\theta}_{ML} = \hat{\theta}_{MVUB} =
\frac{1}{\bm{g'C_z^{-1}g}}\bm{g'C_z^{-1}y}
\begin{array}{c}
	\theta_0 \\
	\gtrless \\
	\theta_1 \\
\end{array}
\frac{\theta_1 + \theta_0}{2}
\] 

\section{Gaussian Noise and Complex Gaussian $\theta$}%
\label{sec:gaussian_noise_and_gaussian_theta_}
The mean of $\theta|\bm{y}$ here is simply the linear MMSE estimate of $\theta$. This makes sense. With both $\bm{Z}$ and $\theta$ having gaussian distributions, $\theta|\bm{y}$ will also be gaussian, and the mean of $\theta|\bm{y}$ will be our best prediction of $\theta$. It also makes sense that $\theta|\bm{y}$ has a complex normal distribution, as  $\theta$ is complex valued. Intuitively, it also makes sense that the covariance of $\theta|\bm{y}$ is proportional to $\bm{C_z}$, and is equal to $\sigma_\theta^2$ in the noiseless case.

\subsection{MMSE}
\[
\hat{\theta}_{MMSE} = E\{\theta|\bm{y}\} \\
\] 
Since $\theta|\bm{y}$ is gaussian, it's expected value is just the mean, which is \\
$$\hat{\theta}_{MMSE} = \frac{\sigma_\theta^2}{1+\sigma_\theta^2 \bm{g'C_z^{-1}g}}\bm{g'C_z^{-1}y}$$

\subsection{ABS}
\[
\hat{\theta}_{ABS} = median(\theta|\bm{y})
\] 
Since $\theta|\bm{y}$ is gaussian, the median is the same as the mean, so once again \\
$$\hat{\theta}_{ABS} = \frac{\sigma_\theta^2}{1+\sigma_\theta^2 \bm{g'C_z^{-1}g}}\bm{g'C_z^{-1}y}$$

\subsection{MAP}
\[
\hat{\theta}_{MAP} = argmax_\theta f(\theta|\bm{y})
\] 
Yet again, the gaussian shape of $\theta|\bm{y}$ makes our job easier. The peak of the distribution is also the mean, so \\
$$\hat{\theta}_{MAP} = \frac{\sigma_\theta^2}{1+\sigma_\theta^2 \bm{g'C_z^{-1}g}}\bm{g'C_z^{-1}y}$$

\end{document}
